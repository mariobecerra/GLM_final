\documentclass{article}
\usepackage[utf8]{inputenc}
\usepackage{blindtext}
\usepackage{graphicx}
\usepackage[pass]{geometry}
\usepackage[backend=bibtex, maxbibnames=99]{biblatex}
\usepackage{parskip}
\usepackage[spanish]{babel}
\usepackage{csquotes}


\addbibresource{bibliography.bib} %Imports bibliography file

\begin{document}

\newgeometry{bottom=2.5cm, left=2.1cm,right=2.1cm,top=2.5cm}

\title{Regresión Avanzada \\ Proyecto Final}

\author{Mario~Becerra Contreras \\ Edgar~López}

\date{Otoño 2017}


\maketitle

%!TEX root = ../GLM_Becerra_Lopez.tex

\section{Introducción}
\label{sec:intro}

En este trabajo se presentan distintos modelos para el precio de casas y departamentos en la ciudad de Nueva York, esto con el objetivo de establecer un rango de precios de venta de una casa en una primera instancia, es decir, poder dar una primera cotización acerca de su precio sin conocer demasiados detalles acerca de ella. Un aspecto importante que se debe considerar es que aunque dos casas cuenten con características similares, sus precios pueden variar de manera sustancial dependiendo su ubicación geográfica, por lo que es importante que la cotización de venta de una casa esté en línea con los precios de la zona.

Para generar un modelo que sea útil a una inmobiliaria o agente de ventas, incluyendo  los aspectos descritos anteriormente, es necesario utilizar técnicas estadísticas que permitan explicar el precio de venta de una casa a partir de una seria de variables que representen características de esta, considerando la variabilidad de precios que existe por su ubicación geográfica.

Los datos fueron obtenidos de Kaggle\footnote{\url{https://www.kaggle.com/new-york-city/nyc-property-sales}}, una plataforma de concursos de modelado predictivo y aprendizaje estadístico. Los datos de Kaggle, a su vez, provinieron de la página oficial del departamento de finanzas de la ciudad de Nueva York\footnote{\url{http://www1.nyc.gov/site/finance/taxes/property-rolling-sales-data.page}}.

Los datos tienen variables relacionadas con las casas y sus ventas, como distrito (\textit{borough}), vecindario (\textit{neighborhood}), código postal (\textit{zip code}), dirección, precio de venta del inmueble, fecha de venta, tipo de inmueble, tamaño del inmueble, etc. No se tiene información muy específica como número de cuartos o de baños.

Se tienen $84,548$ observaciones de un periodo de 12 meses (de septiembre de 2016 a agosto de 2017), las cuales no solo incluyen información de inmuebles residenciales, sino todo tipo de bienes raíces, por lo que se filtraron los datos para obtener solamente las observaciones correspondientes a casas. Además, había muchos datos que tenían como precio de venta 0, lo cual puede ser por herencia de padres a hijos o algún otro tipo de traspaso sin dinero\footnote{\url{http://www1.nyc.gov/assets/finance/downloads/pdf/07pdf/glossary_rsf071607.pdf}}. También existían ventas con valores no creíbles, como unos pocos miles de dólares, por lo que tampoco se tomaron en cuenta para este trabajo; además, también se filtraron las observaciones que no tenían información sobre el código postal. Después de este filtrado, quedaron $25,299$ observaciones.

En las siguientes subsecciones se muestra el análisis exploratorio de los datos.

%!TEX root = ../GLM_Becerra_Lopez.tex

\section{Datos}
\label{sec:datos}

Los datos fueron obtenidos de Kaggle\footnote{\url{https://www.kaggle.com/new-york-city/nyc-property-sales}}, una plataforma de concursos de modelado predictivo y aprendizaje estadístico. Los datos de Kaggle, a su vez, provinieron de la página oficial del departamento de finanzas de la ciudad de Nueva York\footnote{\url{http://www1.nyc.gov/site/finance/taxes/property-rolling-sales-data.page}}.

Los datos tienen variables relacionadas con las casas y sus ventas, como distrito (\textit{borough}), vecindario (\textit{neighborhood}), código postal (\textit{zip code}), dirección, precio de venta del inmueble, fecha de venta, tipo de inmueble, tamaño del inmueble, etc. No se tiene información muy específica como número de cuartos o de baños.

Se tienen $84,548$ observaciones de un periodo de 12 meses (de septiembre de 2016 a agosto de 2017), las cuales no solo incluyen información de inmuebles residenciales, sino todo tipo de bienes raíces, por lo que se filtraron los datos para obtener solamente las observaciones correspondientes a casas. Además, había muchos datos que tenían como precio de venta 0, lo cual puede ser por herencia de padres a hijos o algún otro tipo de traspaso sin dinero\footnote{\url{http://www1.nyc.gov/assets/finance/downloads/pdf/07pdf/glossary_rsf071607.pdf}}. También existían ventas con valores no creíbles, como unos pocos miles de dólares, por lo que tampoco se tomaron en cuenta para este trabajo; además, también se filtraron las observaciones que no tenían información sobre el código postal. Después de este filtrado, quedaron $25,299$ observaciones.

En las siguientes subsecciones se muestra el análisis exploratorio de los datos.

\subsection{Gráficas univariadas}

La principal variable de interés es el precio en dólares de venta de las casas en Nueva York. En la figura \ref{fig:eda_histograma_precio_venta} se muestra una gráfica de frecuencias absoluta con y sin la transformación logaritmo. Como se puede observar, los precios de venta se asemejan a una distribución exponencial o gamma, pero aplicando la transformación logaritmo, los datos se asemejan a una muestra de una distribución normal, por lo que se usará esta transformación en los modelos posteriores.

\begin{figure}[H]
    \centering
    \includegraphics[width=0.9\textwidth]{images/eda_histograma_precio_venta.pdf}
    \caption{Histogramas del precio de venta en escala original y en escala logarítmica}
    \label{fig:eda_histograma_precio_venta}
\end{figure}


Otra de las variables de interés es la superficie total que esta medida en pies cuadrados. La figura \ref{fig:eda_histograma_superficie} muestra las gráficas de frecuencias absolutas para esta variable con y sin transformación logaritmo. De igual manera que el precio de ventas, sería más conveniente usar los datos usando la transformación logaritmo pues muestran un comportamiento semejante a una muestra de una distribución normal.

\begin{figure}[H]
    \centering
    \includegraphics[width=0.9\textwidth]{images/eda_histograma_superficie.pdf}
    \caption{Histogramas de la superficie de construcción en escala original y en escala logarítmica}
    \label{fig:eda_histograma_superficie}
\end{figure}


Finalmente, la variable de superficie del terreno en pies cuadrados se muestra en la figura \ref{fig:eda_histograma_superficie_total_land}. En este caso también sería conveniente usar la transformación logaritmo en los datos pues mejora la distribución muestral y reescala los datos a una escala más pequeña.

\begin{figure}[H]
    \centering
    \includegraphics[width=0.9\textwidth]{images/eda_histograma_superficie_total_land.pdf}
    \caption{Histogramas de la superficie del terreno en escala original y en escala logarítmica}
    \label{fig:eda_histograma_superficie_total_land}
\end{figure}






\subsection{Gráficas bivariadas}

Primero se analizará la posible relación entre el precio de venta y la superficie total mediante un diagrama de dispersión. En la figura \ref{fig:eda_dispersion_superficie_vs_precio} se puede ver la gráfica de dispersión de la superficie de construcción contra el precio. Existe una tendencia lineal creciente entre las dos variables, es decir, a mayor superficie total también se tiene un mayor precio de venta. Por lo que la variable de superficie total puede ser usada como variable explicativa en un modelo de regresión.

\begin{figure}[H]
    \centering
    \includegraphics[width=0.7\textwidth]{images/eda_dispersion_superficie_vs_precio.pdf}
    \caption{Gráfica de dispersión de superficie de construcción contra precio}
    \label{fig:eda_dispersion_superficie_vs_precio}
\end{figure}


La relación entre la superficie de terreno y el precio de venta se muestra en el diagrama de dispersión de la figura \ref{fig:eda_dispersion_superficie_total_vs_precio}. Visualmente no existe una relación entre estas dos variables pues no muestra alguna tendencia. En la figura \ref{fig:eda_dispersion_superficie_total_vs_superficie} se muestra la posible relación entre las covariables superficie de construcción y superficie de terreno. Como es de esperarse, estas dos variables están relacionadas pues se puede esbozar una relación creciente, es decir, a mayor superficie total se tiene mayor superficie. Dada esta colinealidad, en un modelo de regresión se debería de usar alguna de estas dos variables pues proporcionan la misma información. En el caso de este trabajo, se seleccionó la variable de superficie de construcción debido a su correlación con el precio.


\begin{figure}[H]
    \centering
    \includegraphics[width=0.7\textwidth]{images/eda_dispersion_superficie_total_vs_precio.pdf}
    \caption{Gráfica de dispersión de superficie de terreno contra precio}
    \label{fig:eda_dispersion_superficie_total_vs_precio}
\end{figure}


\begin{figure}[H]
    \centering
    \includegraphics[width=0.7\textwidth]{images/eda_dispersion_superficie_total_vs_superficie.pdf}
    \caption{Gráfica de dispersión de superficie de construcción contra superficie del terreno}
    \label{fig:eda_dispersion_superficie_total_vs_superficie}
\end{figure}


Es de esperar que el precio de venta cambie dependiendo si la casa esta ubicada en cierto distrito (\textit{borough}). Para corroborar esta hipótesis se graficaron los precios en cada uno de los distritos, los cuales se pueden ver en las figuras \ref{fig:eda_histogram_price_borough} y \ref{fig:eda_boxplot_price_borough}. Se puede ver que, en efecto, cambian las distribuciones muestrales dependiendo el distrito en el que se encuentran las casas. También se puede ver que Manhattan muestra una media más alta que el resto de los distritos (línea punteada en figura \ref{fig:eda_boxplot_price_borough}), y también presenta más variación en los precios de venta. El siguiente distrito con una media más alta es Brooklyn con una variación más grande que el resto de los distritos (sin considerar Manhattan). 

\begin{figure}[H]
    \centering
    \includegraphics[width=0.7\textwidth]{images/eda_histogram_price_borough.pdf}
    \caption{Histogramas de precio de venta por distrito}
    \label{fig:eda_histogram_price_borough}
\end{figure}


\begin{figure}[H]
    \centering
    \includegraphics[width=0.7\textwidth]{images/eda_boxplot_price_borough.pdf}
    \caption{Diagrama de caja y brazos de precio de venta por distrito}
    \label{fig:eda_boxplot_price_borough}
\end{figure}

 

Hasta este momento se ha omitido la variable vecindario en el análisis exploratorio. La figura \ref{fig:eda_scatter_by_neighborhood} muestra la dispersión de los datos en cada vecindario. Se puede observar que considerando los vecindarios dentro de cada distrito, en algunos la relación creciente no es tan clara, o incluso llega a verse decreciente, como en el Upper West Side de Manhattan. Hay que tomar en cuenta que el número de observaciones en este vecindario es más pequeño. %Por ejemplo, en el distrito de Queens, los vecindarios Southeast Queens y Rockaways. Sin embargo los datos parecen seguir agrupandose por el tipo de casa que es. Tambien se puede observar que no en todas los vecindarios se encuentran todos los tipos de casas.

El propósito de esta gráfica es mostrar que las relaciones cambian de vecindario a vecindario, por lo que hay que tomar esto en cuenta al momento de hacer los modelos. Aún cuando existe otro nivel geográfico (los códigos postales), no es factible graficar la relación a este nivel pues son demasiados para que se pueda ver en una hoja, sin embargo, se hizo un análisis y también se ve que las relaciones cambian.

\begin{figure}[H]
    \centering
    \includegraphics[width=1.03\textwidth]{images/eda_scatter_by_neighborhood.pdf}
    \caption{Diagramas de dispersión del logaritmo del precio contra el logaritmo del tamaño en cada vecindario}
    \label{fig:eda_scatter_by_neighborhood}
\end{figure}



%!TEX root = ../GLM_Becerra_Lopez.tex

\section{Modelos estadísticos}
\label{sec:modelos}

El objetivo es obtener estimaciones del precio de las casas a partir del tamaño en pies cuadrados. Para probar la capacidad predictiva, se dividieron los datos en dos: un conjunto de entrenamiento con el 90\% de los datos, y un conjunto de prueba con el 10\% restante. Para tener información de todos los códigos postales, se hizo un muestreo estratificado, tomando el 90\% de observaciones de cada código postal para el conjunto de entrenamiento. En el conjunto de entrenamiento quedaron $22,769$ observaciones y en el de prueba $2,530$.	

Se ajustaron tres modelos lineales a los datos: un modelo de unidades iguales, un modelo de unidades independientes, y un modelo jerárquico. El modelo de unidades iguales asume que todas las realizaciones provienen de la misma distribución; mientras que el de unidades independientes asume que los precios varían de acuerdo a diferentes sectores (en este caso son los códigos postales); y finalmente, el modelo jerárquico es un compromiso entre ambos modelos que toma fuerza de los demás sectores, esto es particularmente útil cuando hay sectores con pocas observaciones.

\subsection{Modelo de unidades iguales}

El modelo de unidades iguales es simplemente un modelo de regresión lineal con un parámetro fijo para el intercepto y un parámetro fijo para cada uno de los regresores. Sean $y_i$ el logaritmo del precio de la casa $i$ y $x_i$ el logaritmo del número de pies cuadrados en la casa $i$, para $i \in \left\{1, \hdots, n \right\}$, con $n = 22,769$. El modelo de unidades independientes es $y_i \sim \mathrm{N}(\alpha + \beta x_i, \tau_y)$, con distribuciones previas $\alpha \sim \mathrm{N}(0, 0.001)$, $\beta \sim \mathrm{N}(0, 0.001)$ y $\tau_y \sim \mathrm{Ga}(0.001, 0.001)$. %$\alpha \sim \mathrm{N}(\alpha_0, \tau_{\alpha})$, $\beta \sim \mathrm{N}(\beta_0, \tau_{\beta})$ y $\tau_y \sim \mathrm{Ga}(a, b)$

De antemano se tiene conocimiento como para pensar que este modelo no es el más adecuado para los datos, pues se vio en el análisis exploratorio de datos que los precios varían por código postal, por lo que no es muy sensato suponer que no existen relaciones entre las observaciones.

De hecho, en la figura \ref{fig:comp_pooling_resids} se puede ver este efecto. En la figura se muestran los residuales del modelo en el eje $y$, y en el eje $x$ se muestra el índice de la observación, donde las observaciones están ordenadas de acuerdo a código postal. Es evidente un patrón, que viene de la correlación entre las observaciones que existe dentro de cada código postal.

\begin{figure}[H]
    \centering
    \includegraphics[width=0.9\textwidth]{images/comp_pooling_resids.pdf}
    \caption{Residuales de modelo de unidades iguales}
    \label{fig:comp_pooling_resids}
\end{figure}

En la figura \ref{fig:comp_pooling_obs_vs_pred} se puede ver para cada observación el valor observado contra el valor ajustado. Se aprecia una varianza considerablemente grande y que además en los valores pequeños, el modelo tiende a sobreestimar, mientras que en valores grandes pasa lo contrario.

\begin{figure}[H]
    \centering
    \includegraphics[width=0.9\textwidth]{images/comp_pooling_obs_vs_pred.pdf}
    \caption{Ajustado contra observado en modelo de unidades iguales}
    \label{fig:comp_pooling_obs_vs_pred}
\end{figure}

\subsection{Modelo de unidades independientes}

Este es un modelo de interceptos y pendientes cambiantes de acuerdo al código postal, es decir, es de la forma $y_i \sim \mathrm{N}(\alpha_{j[i]} + \beta_{j[i]}, \tau_y)$, donde $j[i]$ se refiere al código postal correspondiente a la $i$-ésima observación. Las distribuciones previas son $\alpha[j] \sim \mathrm{N}(0, 0.001)$, $\beta[j] \sim \mathrm{N}(0, 0.001)$ y $\tau_y \sim \mathrm{Ga}(0.001, 0.001)$, para $j = 1, \hdots, J$, con $J = 154$ el número de códigos postales en la ciudad.

\begin{figure}[H]
    \centering
    \includegraphics[width=0.9\textwidth]{images/no_pooling_resids.pdf}
    \caption{Residuales de modelo de unidades independientes}
    \label{fig:no_pooling_resids}
\end{figure}

\begin{figure}[H]
    \centering
    \includegraphics[width=0.9\textwidth]{images/no_pooling_obs_vs_pred.pdf}
    \caption{Valor ajustado contra observado en modelo de unidades independientes}
    \label{fig:no_pooling_obs_vs_pred}
\end{figure}

\begin{figure}[H]
    \centering
    \includegraphics[width=0.9\textwidth]{images/no_pooling_param_values.pdf}
    \caption{Valor e intervalos de probabilidad de parámetros de modelo de unidades independientes}
    \label{fig:no_pooling_param_values}
\end{figure}


\subsection{Modelo multinivel}



\begin{figure}[H]
    \centering
    \includegraphics[width=0.9\textwidth]{images/three_levels_resids.pdf}
    \caption{Residuales de modelo multinivel}
    \label{fig:three_levels_resids}
\end{figure}

\begin{figure}[H]
    \centering
    \includegraphics[width=0.9\textwidth]{images/three_levels_obs_vs_pred.pdf}
    \caption{Valor ajustado contra observado en modelo multinivel}
    \label{fig:three_levels_obs_vs_pred}
\end{figure}

\begin{figure}[H]
    \centering
    \includegraphics[width=0.9\textwidth]{images/three_levels_param_values.pdf}
    \caption{Valor e intervalos de probabilidad de parámetros de modelo multinivel}
    \label{fig:three_levels_param_values}
\end{figure}


%!TEX root = ../GLM_Becerra_Lopez.tex

\section{Resultados}
\label{sec:resultados}

En la figura \ref{fig:comp_pooling_resids} se muestran los residuales del modelo en el eje $y$, y en el eje $x$ se muestra el índice de la observación, donde las observaciones están ordenadas de acuerdo a código postal. Es evidente un patrón, que viene de la correlación entre las observaciones que existe dentro de cada código postal.

\begin{figure}[H]
    \centering
    \includegraphics[width=0.8\textwidth]{images/comp_pooling_resids.pdf}
    \caption{Residuales de modelo de unidades iguales}
    \label{fig:comp_pooling_resids}
\end{figure}

En la figura \ref{fig:comp_pooling_obs_vs_pred} se puede ver para cada observación el valor observado contra el valor ajustado. Se aprecia una varianza considerablemente grande y que además en los valores pequeños, el modelo tiende a sobreestimar, mientras que en valores grandes pasa lo contrario.

\begin{figure}[H]
    \centering
    \includegraphics[width=0.8\textwidth]{images/comp_pooling_obs_vs_pred.pdf}
    \caption{Ajustado contra observado en modelo de unidades iguales}
    \label{fig:comp_pooling_obs_vs_pred}
\end{figure}


En la figura \ref{fig:no_pooling_resids} se pueden ver los residuales del modelo de unidades independientes. Se puede ver que ya no se ven los patrones tan evidentes del código postal como en el modelo de unidades iguales.

\begin{figure}[H]
    \centering
    \includegraphics[width=0.8\textwidth]{images/no_pooling_resids.pdf}
    \caption{Residuales de modelo de unidades independientes}
    \label{fig:no_pooling_resids}
\end{figure}

En la figura \ref{fig:no_pooling_obs_vs_pred} se muestra el valor observado contra el valor ajustado

\begin{figure}[H]
    \centering
    \includegraphics[width=0.8\textwidth]{images/no_pooling_obs_vs_pred.pdf}
    \caption{Valor ajustado contra observado en modelo de unidades independientes}
    \label{fig:no_pooling_obs_vs_pred}
\end{figure}

\begin{figure}[H]
    \centering
    \includegraphics[width=0.9\textwidth]{images/no_pooling_param_values.pdf}
    \caption{Valor e intervalos de probabilidad de parámetros de modelo de unidades independientes}
    \label{fig:no_pooling_param_values}
\end{figure}




En la figura \ref{fig:three_levels_resids} se pueden ver los residuales del modelo multinivel.


\begin{figure}[H]
    \centering
    \includegraphics[width=0.8\textwidth]{images/three_levels_resids.pdf}
    \caption{Residuales de modelo multinivel}
    \label{fig:three_levels_resids}
\end{figure}

\begin{figure}[H]
    \centering
    \includegraphics[width=0.8\textwidth]{images/three_levels_obs_vs_pred.pdf}
    \caption{Valor ajustado contra observado en modelo multinivel}
    \label{fig:three_levels_obs_vs_pred}
\end{figure}

\begin{figure}[H]
    \centering
    \includegraphics[width=0.9\textwidth]{images/three_levels_param_values.pdf}
    \caption{Valor e intervalos de probabilidad de parámetros de modelo multinivel}
    \label{fig:three_levels_param_values}
\end{figure}

\subsection{Comparación de modelos}

En esta subsección se comparan los modelos.


% Gráficas del diagnóstico de Gelman para convergencia
% Comparar tamaño de muestra efectivo vs obtenido
% https://cran.r-project.org/web/packages/bayesplot/vignettes/visual-mcmc-diagnostics.html

% Probar con distintas priors y ver si resultados son los mismos

% Comparar coeficientes y promedios en distintos boroughs, vecindarios y zip codes
% Ver RMSE por borough y vecindario para cada modelo

% Mapa con valores de predicción junto con su correspondiente incertidumbre (la mitad de la longitud del intervalo de 95% de credibilidad)

% Comparar modelo final con modelo de interceptos distintos por zip sin jerearquía
% Comparar vs modelo nulo

% Analizar algunos outliers del modelo

%!TEX root = ../GLM_Becerra_Lopez.tex

\section{Conclusiones}
\label{sec:conclusiones}

Los tres modelos presentados anteriormente son soluciones a un mismo problema: Predecir el precio de una casa en venta. El modelo de unidades iguales no considera la variabilidad de los precios dependiendo en la zona geográfica, lo que lo hace un modelo poco creíble desde su definición. El modelo de unidades independientes incorpora esta variabilidad pero no la capta en su totalidad, y aunque en algunas métricas de evaluación de ajuste salga mejor que los demás modelos, algunos de los resultados arrojados por el modelo no son lógicos. Finalmente, el modelo multinivel también incorpora la variabilidad de los precios pero a distintos niveles geográficos por lo es un modelo más robusto, otra característica importante es que a diferencia de los modelos anteriores, utiliza toda la información disponible. Por lo que se concluye que el modelo multinivel es el mejor modelo para responder al problema planteado.

Es importante mencionar que aún cuando un modelo presenta mejores métricas de ajuste, no necesariamente es el que tiene mejor interpretabilidad, por lo que en la práctica estadística es recomendable comparar los resultados de un modelo contra la realidad.

Asimismo, en este trabajo también se puede observar otra de las aplicaciones de los modelos de regresión: estimación de datos faltantes. Pues mediante el modelo multinivel se realizó una imputación de datos faltantes, con base en su ubicación geográfica, a aquellas casas que no tenían un precio de venta.


\printbibliography
\nocite{*}


\end{document}
