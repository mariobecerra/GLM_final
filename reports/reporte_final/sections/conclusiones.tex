%!TEX root = ../GLM_Becerra_Lopez.tex

\section{Conclusiones}
\label{sec:conclusiones}

Los tres modelos presentados anteriormente son soluciones a un mismo problema: predecir el precio de una casa en venta. El modelo de unidades iguales no considera la variabilidad de los precios dependiendo en la zona geográfica, lo que lo hace un modelo poco creíble desde su definición. El modelo de unidades independientes incorpora esta variabilidad pero no la capta en su totalidad, y aunque en algunas métricas de evaluación de ajuste salga mejor que los demás modelos, algunos de los resultados arrojados por el modelo no son lógicos. Finalmente, el modelo multinivel también incorpora la variabilidad de los precios pero a distintos niveles geográficos por lo es un modelo más robusto. Otra característica importante es que a diferencia de los otros modelos, el modelo multinivel utiliza toda la información disponible. Por lo que se concluye que el modelo multinivel es el mejor modelo para responder al problema planteado.

Es importante mencionar que aún cuando un modelo presenta mejores métricas de ajuste, no necesariamente es el que tiene mejor interpretabilidad, por lo que en la práctica estadística es recomendable comparar los resultados de un modelo contra la realidad. También creemos que las métricas de ajuste pueden mejorar con más covariables que describan las casas, por ejemplo, número de cuartos, número de baños, estado de la casa, etc. Sin embargo, en este ejercicio estas variables no estaban disponibles.

Asimismo, en este trabajo también se puede observar otra de las aplicaciones de los modelos de regresión: estimación de datos faltantes. Pues mediante el modelo multinivel se realizó una estimación de cual sería el precio de venta de un pie cudrado en códigos postales que no registrarón ventas.
